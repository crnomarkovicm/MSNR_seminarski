\documentclass[a4paper]{article}
\usepackage[table]{xcolor}
\usepackage{color}
\usepackage{url}
\usepackage{pgf-pie}
\usepackage[T2A]{fontenc} % enable Cyrillic fonts
\usepackage[utf8]{inputenc} % make weird characters work
\usepackage{graphicx}

\usepackage[english,serbian]{babel}
%\usepackage[english,serbianc]{babel} %ukljuciti babel sa ovim opcijama, umesto gornjim, ukoliko se koristi cirilica

\usepackage[unicode]{hyperref}
\hypersetup{colorlinks,citecolor=green,filecolor=green,linkcolor=blue,urlcolor=blue}

\usepackage{listings}

%\newtheorem{primer}{Пример}[section] %ćirilični primer
\newtheorem{primer}{Primer}[section]

\definecolor{mygreen}{rgb}{0,0.6,0}
\definecolor{mygray}{rgb}{0.5,0.5,0.5}
\definecolor{mymauve}{rgb}{0.58,0,0.82}

\lstset{ 
  backgroundcolor=\color{white},   % choose the background color; you must add \usepackage{color} or \usepackage{xcolor}; should come as last argument
  basicstyle=\scriptsize\ttfamily,        % the size of the fonts that are used for the code
  breakatwhitespace=false,         % sets if automatic breaks should only happen at whitespace
  breaklines=true,                 % sets automatic line breaking
  captionpos=b,                    % sets the caption-position to bottom
  commentstyle=\color{mygreen},    % comment style
  deletekeywords={...},            % if you want to delete keywords from the given language
  escapeinside={\%*}{*)},          % if you want to add LaTeX within your code
  extendedchars=true,              % lets you use non-ASCII characters; for 8-bits encodings only, does not work with UTF-8
  firstnumber=1000,                % start line enumeration with line 1000
  frame=single,	                   % adds a frame around the code
  keepspaces=true,                 % keeps spaces in text, useful for keeping indentation of code (possibly needs columns=flexible)
  keywordstyle=\color{blue},       % keyword style
  language=Python,                 % the language of the code
  morekeywords={*,...},            % if you want to add more keywords to the set
  numbers=left,                    % where to put the line-numbers; possible values are (none, left, right)
  numbersep=5pt,                   % how far the line-numbers are from the code
  numberstyle=\tiny\color{mygray}, % the style that is used for the line-numbers
  rulecolor=\color{black},         % if not set, the frame-color may be changed on line-breaks within not-black text (e.g. comments (green here))
  showspaces=false,                % show spaces everywhere adding particular underscores; it overrides 'showstringspaces'
  showstringspaces=false,          % underline spaces within strings only
  showtabs=false,                  % show tabs within strings adding particular underscores
  stepnumber=2,                    % the step between two line-numbers. If it's 1, each line will be numbered
  stringstyle=\color{mymauve},     % string literal style
  tabsize=2,	                   % sets default tabsize to 2 spaces
  title=\lstname                   % show the filename of files included with \lstinputlisting; also try caption instead of title
}

\begin{document}

\title{Problemi rodne ravnopravnosti u informatici u Srbiji\\
\vspace{5mm}
\small{Seminarski rad u okviru kursa\\Metodologija stručnog i naučnog rada\\ Matematički fakultet}}

\author{Marko Babić, Miloš Milaković,  Maja Crnomarković\\ \href{mailto:markobabic056@gmail.com}{markobabic056@gmail.com} , \href{mailto:milos.milakovic98@gmail.com}{milos.milakovic98gmail.com}, \href{mailto:majacrnomarkovic@live.com}{majacrnomarkovic@live.com}}

%\date{9.~april 2015.}
\maketitle

\abstract{
TODO.}

\tableofcontents

\newpage

\section{Uvod}
\label{sec:uvod}

Ada Bajron, ćerka pesnika Bajrona i talentovani matematičar smatra se prvim programerom u istoriji. Njoj u čast nazvan je programski jezik ADA. Ona je bila i prva koja je uvidela da se računske mašine mogu upotrebiti i za nematematičke namene,
čime je na neki način predvidela današnje primene digitalnih računara. S obzirom na to da je u to vreme bilo nezamislivo da se žene bave bilo kakvim naučnim radom, prevazilazeći stereotipne granice, Ada je naslutila većinu onoga što se danas smatra modernim računarstvom.
Ipak, slika osobe koja se bavi informacionim tehnologijama ili programera oduvek je podrazumevala muškarca. Još od početka razvoja računara, uloga žena bila je da reklamiraju    i takav pristup doprineo je da se i danas žene slabije odlučuju na ove profesije. Tehnologija neizostavno prožima živote svih nas, a vizija ravnopravnog i otvorenog društva podrazumeva da svi, bez obzira na rod, imaju pristup inovacijama, kao i obrazovanju u svim oblastima – kako radi što boljeg razumevanja sveta budućnosti, tako i zbog doprinosa koji mogu pružiti.

\section{Odnos polova u industriji u informatici u Srbiji}


Postoji nekoliko faktora koji doprinose proširenju rodnog jaza u IT sektoru:
1. Ne samo da IT kompanije zapošljavaju manje žena nego muškaraca, već  je i procenat žena koje napuštaju IT poslove veći od procenata muškaraca. (Ashcraft,  McLain  and  Eger,  2016;  Scott  and  Kapor  Klein,  2017)
2. U godinama koje slede, automatizacija će značajno uticati na tržište poslova. Procenjeno je da će u naredne dve decenije ona zameniti 10p poslova. Istraživači u Međunarodnom monetarnom fondu analizirali su trideset država i predviđaju disproporcionalno veći gubitak poslova kod žena nego kod muškaraca. Razlog tome je što će automatizacija najviše uticati na pretežno administrativne poslove kojima se češće bave žene (Brussevich et al., 2018). 



\section{Odnos polova u akademiji u informatici u Srbiji}	
Problem rodne jednakosti, tj. nejednakosti je deo kulturnog nasleđa i teško je uticati na tradicionalne poglede i podelu poslova na osnovu polova. Veliku odgovornost u tome da zainteresuju žene za IT oblast kao i da im omoguće kvalitetno obrazovanje imaju sve obrazovne institucije, a posebno Univerziteti. Bez toga, trenutna slika IT sektora teško će moći da se promeni i unapredi.

Prema podacima Republičkog zavoda za statistiku za 2019. godinu, na visoke škole i fakultete upisano je ukupno 241968 studenata. Na visoko obrazovanje upisalo se 137910 žena, tj. 57\%, dok je muškaraca  104058, tj. 43\%. Sličan odnos bio je i u godinama od 2014. do 2019. (tabela 1).

\setlength{\arrayrulewidth}{0.5mm}
\setlength{\tabcolsep}{14pt}
\renewcommand{\arraystretch}{2.5}
\newcolumntype{m}{>{\columncolor[HTML]{AAACED}} p{1cm}}
\newcolumntype{z}{>{\columncolor[HTML]{FFB6C1}} p{1cm}}
{
\centering
    \begin{tabular}{ |p{1cm}|z|m|z|m| }
\hline
Godina& Žene & Muškarci & Žene (\%) & Muškarci (\%) \\
\hline
2014 &134460 &106594 &55,8 &44,2 \\
2015 &138971 &112191 &55,3 &44,7 \\
2016 &146899 &115190 &56,0 &44,0 \\
2017 &144871 &111301 &56,6 &43,4 \\
2018 &141679 &108092 &56,7 &43,3 \\
2019 &137910 &6104058 &57,0 &43,3 \\
\hline
\end{tabular} \ \
} \\

Veći broj žena odlučuje se za studije u oblastima obrazovanja, zdravstva i zaštite, kao i umetnosti i humanističkim naukama. (slika 2). 
{
\centering
\begin{figure}[h!]
\begin{center}
\includegraphics[scale=0.7]{upisaniNaVisokoObrazovanjeMZ.png}
\end{center}
\caption{Upisani}
\label{fig:upisani}
\end{figure}
}
Među studentima koji su diplomirali 2019. godine, žene čine više od polovine u velikom broju područja obrazovanja, ali su muškarci i dalje dominantni u područjima: Informacione i komunikacione tehnologije (66\%) i Inženjerstvo, proizvodnja i građevinarstvo (61\%). \\

{
\centering
\begin{figure}[h!]
\begin{center}
\includegraphics[scale=0.7]{diplomiraniNaVisokimSkIF.png}
\end{center}
\caption{Diplomirani}
\label{fig:diplomirani}
\end{figure}
}


U Srbiji je 2019. godine doktoriralo 792 osobe, 448 žena i 344 muškarca. 
Iako je veći broj žena koje završavaju doktorske studije (57\%),  u oblasti informatike i komunikacionih tehnolologija taj broj je najmanji (34\%), dok muškarci čine 66\% doktoriranih lica u ovoj oblasti. \\


{
\centering
\begin{figure}[h!]
\begin{center}
\includegraphics[scale=0.7]{doktorirali.png}
\end{center}
\caption{Doktorirali}
\label{fig:doktorirali}
\end{figure}
}

\section{Odnos polova u informatici u Srbiji u odnosu na svet}	

\section{Odnos polova u informatici na Matematičkom fakultetu Univerziteta u Beogradu}
\label{matematicki}

U ovom odeljku prikazaćemo podatke o odnosu polova na smeru Informatika na Matematičkom fakultetu Univerziteta u Beogradu u prethodnih šest godina. \\

{
\centering
\begin{table}[h!]
\caption{Broj upisanih studenata muškog i ženskog pola u prethodnih šest godina}

\begin{tabular}{|p{2cm}|p{2cm}|p{2cm}|p{2cm}|} \hline
godina&broj upisanih muškog pola& broj upisanih ženskog pola & ukupan broj upisanih\\ \hline
2017&144&47&161\\ \hline
2018&102&54&156\\ \hline
2019&90&75&165\\ \hline
2022&104&56&160\\ \hline


\end{tabular}
\label{tab:tabela1}
\end{table}
}

\begin{tikzpicture}
\pie{70.8/M, 29.2/Ž}
\end{tikzpicture}

\begin{tikzpicture}
\pie{65.38/M, 34.62/Ž}
\end{tikzpicture}

\begin{tikzpicture}
\pie{54.55/M, 45.45/Ž}
\end{tikzpicture}

\begin{tikzpicture}
\pie{65/M, 35/Ž}
\end{tikzpicture}

\newpage


\section{Organizacije i inicijative koje se bave rodnom ravnopravnošću}

Neke od organizacija koje između ostalog deluju na teritoriji Srbije:
\begin{itemize}
\item Care International
\item Fondacija BFPE za odgovorno drustvo
\item UNESCO
\item CEDEM
\item Akademija ženskog liderstva
\item EUGAIN
\end{itemize}
Neke od najpoznatijih organizacija koje deluju na globalnom nivou su:

\begin{itemize}
\item UN Women;
\item European Institute for Gender Equality;
\item Global Fund for Women;
\item International Center for Research on Women;
\item Instituto Promundo;
\item Association for Women's Rights in Development;
\item UNESCO
\item Equality Now;
\item National Women's Law Center.
\end{itemize}
Neke od inicijativa u našoj zemlji:

\begin{itemize}
\item Care International - Young Men Initiative (Inicijativa mladića)
\item UNESCO i Globalna alijansa za medije - Žene stvaraju vesti 2015.
\item CEDEM u saradnji sa Crnogorskim ženskim lobijem - Inicijativa za osnaživanje žena: edukacijom za rodnu ravnopravnost
\end{itemize}
Glavni cilj ovih inicijativa je postizanje rodne ravnopravnosti u svim sferama društva.

\section{EUGAIN}

\begin{figure}[h]
\centerline{\includegraphics[scale=0.15]{images/eugain.png}}
\caption{EUGAIN logo}
\end{figure}

\textbf{EUGAIN} je Evrposka mreža za rodnu ravnopravnost u informatici. \\

\ident Žene su nedovoljno zastupljene u oblasti informatike (računarstvu, računarskim naukama, softverskom inženjerstvu...) na svim nivoima, od osnovnih i postdiplomskih studija do učešća i liderstva u akademskoj zajednici i industriji. Povećanje zastupljenosti žena u ovoj oblasti veliki je izazov za akademike, kreatore politike i za društvo u celini. Iako je problem očigledan, napredak je uvek bio spor, uprkos svom trudu i velikoj želji za promene koje se dešavaju širom Evrope. Glavni cilj \textbf{COST akcije} Evropske mreže za rodnu ravnopravnost u informatici (EUGAIN-a) je poboljšanje rodne ravnopravnosti u informatici kroz stvaranje i jačanje zaista multikulturalne evropske mreže akademika koji se nalaze u prvom planu napora u svojim zemljama, institucijama i istraživačkim zajednicama. EUGAIN se zasniva na njihovom znanju, iskustvu, borbama, uspesima i neuspesima, učenju i deljenju svega što je uspelo i kako bi se isto to moglo preneti na druge institucije i države. 
\ident Između ostalih rezultata, EUGAIN će akademskoj zajednici, industriji i drugim zainteresovanim stranama pružiti preporuke i smernice za rešavanje sledećih ključnih izazova:

\begin{enumerate}
\item Kako da više devojaka izabere informatiku kao svoje višeškolske studije i profesiju;

\item Kako zadržati studentkinje i osigurati da završe studije i započnu uspešnu karijeru u ovoj oblasti;

\item Kako podstaći više žena doktora nauka i postdoktorskih istraživača da ostanu u akademskoj karijeri i konkurišu za profesora na odsecima za informatiku;

\item Kako podžati i inspirisati mlade žene u njihovim karijerama i pomoći im da prevaziđu glavne prepreke koje sprečavaju žene da dođu do visokih pozicija.
\end{enumerate}
\ident Gore smo pomenuli COST akciju pa hajde da malo detaljnije objasnimo šta ona podrazumeva:

\ident COST (European Cooperation In Science & Technology) je program finansiranja Evropske Unije koji omogućava istraživačima da uspostave svoje interdisciplinarne istraživačke mreže u Evropi i šire. EUGAIN obezbeđuje sredstva za organizovanje sastanaka, konferencija, kratkih naučnih razmena ili drugih aktivnosti umrežavanja u širokom spektru naučnih tema. Stvaranjem otvorenih prostora gde ljudi i ideje mogu da rastu EUGAIN otključava pun potencijal nauke.
Prioritet svake COST akcije je unapređenje istraživačke saradnje u kojoj ideje u ljudi mogu rasti bez granica. 

\ident \textbf{ITC (Inclusiveness Target Countries)} konferencijske stipendije pružaju finansijsku podršku mladim istraživačima i inovatorima koji su povezani u jednoj od ITC ili bliskoj susednoj državi za njihovo učešće na konferenciji visokog nivoa. Dobitnik stipendije dobija podršku za prisustvo i predstavljanje svog rada na konferenciji i može uspostaviti nove konekcije za buduću saradnju. Možete se prijaviti za ITC konferencijsku stipendiju ako:
\begin{enumerate}
\item ste student doktorskih studija ili istraživač u ranoj karijeri (do 8 godina od dana kada ste stekli doktorat);

\item vaša primarna povezanost je sa institucijom koja se nalazi u jednoj od država koje spadaju u ICT, a to su: Srbija, Albanija, Bosna i Hercegovina, Hrvatska, Bugarska, Kipar, Češka Republika, Letonija, Estonija, Litvanija, Malta, Republika Moldavija, Crna Gora, Gruzija, Grčka, Mađarska, Makedonija, Portugal, Poljska, Rumunija, Slovačka, Slovenija, Turska i Ukrajna.

\item prihvaćeni ste da predstavite rad i navedeni ste uzvaničnom programu konferencije. Morate priznati COST u svom doprinosu.
\end{enumerate}
EUGAIN ima ogroman značaj u postizanju rodne ravnopravnosti u informatici kako u državama Evropske Unije tako i u ostalim evropskim državama. Kako je Srbija jedna od država koje spadaju u ICT mnogim zainteresovanim akademicima je omogućeno da prate ili učestvuju u konferencijama koje organizuje EUGAIN, a koje same za glavni cilj imaju postizanje rodne ravnopravnosti u informatici.

\section{Zaključak}
\label{sec:zakljucak}

TODO

\addcontentsline{toc}{section}{Literatura}
\appendix
\bibliography{seminarski} 
\bibliographystyle{plain}

\appendix


\end{document}

