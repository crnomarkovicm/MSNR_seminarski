\documentclass[a4paper]{article}

\usepackage{color}
\usepackage{url}
\usepackage{pgf-pie}
\usepackage[T2A]{fontenc} % enable Cyrillic fonts
\usepackage[utf8]{inputenc} % make weird characters work
\usepackage{graphicx}

\usepackage[english,serbian]{babel}
%\usepackage[english,serbianc]{babel} %ukljuciti babel sa ovim opcijama, umesto gornjim, ukoliko se koristi cirilica

\usepackage[unicode]{hyperref}
\hypersetup{colorlinks,citecolor=green,filecolor=green,linkcolor=blue,urlcolor=blue}

\usepackage{listings}

%\newtheorem{primer}{Пример}[section] %ćirilični primer
\newtheorem{primer}{Primer}[section]

\definecolor{mygreen}{rgb}{0,0.6,0}
\definecolor{mygray}{rgb}{0.5,0.5,0.5}
\definecolor{mymauve}{rgb}{0.58,0,0.82}

\lstset{ 
  backgroundcolor=\color{white},   % choose the background color; you must add \usepackage{color} or \usepackage{xcolor}; should come as last argument
  basicstyle=\scriptsize\ttfamily,        % the size of the fonts that are used for the code
  breakatwhitespace=false,         % sets if automatic breaks should only happen at whitespace
  breaklines=true,                 % sets automatic line breaking
  captionpos=b,                    % sets the caption-position to bottom
  commentstyle=\color{mygreen},    % comment style
  deletekeywords={...},            % if you want to delete keywords from the given language
  escapeinside={\%*}{*)},          % if you want to add LaTeX within your code
  extendedchars=true,              % lets you use non-ASCII characters; for 8-bits encodings only, does not work with UTF-8
  firstnumber=1000,                % start line enumeration with line 1000
  frame=single,	                   % adds a frame around the code
  keepspaces=true,                 % keeps spaces in text, useful for keeping indentation of code (possibly needs columns=flexible)
  keywordstyle=\color{blue},       % keyword style
  language=Python,                 % the language of the code
  morekeywords={*,...},            % if you want to add more keywords to the set
  numbers=left,                    % where to put the line-numbers; possible values are (none, left, right)
  numbersep=5pt,                   % how far the line-numbers are from the code
  numberstyle=\tiny\color{mygray}, % the style that is used for the line-numbers
  rulecolor=\color{black},         % if not set, the frame-color may be changed on line-breaks within not-black text (e.g. comments (green here))
  showspaces=false,                % show spaces everywhere adding particular underscores; it overrides 'showstringspaces'
  showstringspaces=false,          % underline spaces within strings only
  showtabs=false,                  % show tabs within strings adding particular underscores
  stepnumber=2,                    % the step between two line-numbers. If it's 1, each line will be numbered
  stringstyle=\color{mymauve},     % string literal style
  tabsize=2,	                   % sets default tabsize to 2 spaces
  title=\lstname                   % show the filename of files included with \lstinputlisting; also try caption instead of title
}

\begin{document}

\title{Problemi rodne ravnopravnosti u informatici u Srbiji\\
\vspace{5mm}
\small{Seminarski rad u okviru kursa\\Metodologija stručnog i naučnog rada\\ Matematički fakultet}}

\author{Marko Babić, Miloš Milaković,  Maja Crnomarković\\ \href{mailto:markobabic056@gmail.com}{markobabic056@gmail.com} , \href{mailto:milos.milakovic98@gmail.com}{milos.milakovic98gmail.com}, \href{mailto:majacrnomarkovic@live.com}{majacrnomarkovic@live.com}}

%\date{9.~april 2015.}
\maketitle

\abstract{
TODO.}

\tableofcontents

\newpage

\section{Uvod}
\label{sec:uvod}

TODO

\section{Odnos polova u industriji u informatici u Srbiji}
Ada Bajron, ćerka pesnika Bajrona i talentovani matematičar smatra se prvim programerom u istoriji. Njoj u čast nazvan je programski jezik ADA. Ona je bila i prva koja je uvidela da se računske mašine mogu upotrebiti i za nematematičke namene,
čime je na neki način predvidela današnje primene digitalnih računara. S obzirom na to da je u to vreme bilo nezamislivo da se žene bave bilo kakvim naučnim radom, prevazilazeći stereotipne granice, Ada je naslutila većinu onoga što se danas smatra modernim računarstvom.
Ipak, slika osobe koja se bavi informacionim tehnologijama ili programera oduvek je podrazumevala muškarca. Još od početka razvoja računara, uloga žena bila je da reklamiraju    i takav pristup doprineo je da se i danas žene slabije odlučuju na ove profesije. Tehnologija neizostavno prožima živote svih nas, a vizija ravnopravnog i otvorenog društva podrazumeva da svi, bez obzira na rod, imaju pristup inovacijama, kao i obrazovanju u svim oblastima – kako radi što boljeg razumevanja sveta budućnosti, tako i zbog doprinosa koji mogu pružiti.

Postoji nekoliko faktora koji doprinose proširenju rodnog jaza u IT sektoru:
1. Ne samo da IT kompanije zapošljavaju manje žena nego muškaraca, već  je i procenat žena koje napuštaju IT poslove veći od procenata muškaraca. (Ashcraft,  McLain  and  Eger,  2016;  Scott  and  Kapor  Klein,  2017)
2. U godinama koje slede, automatizacija će značajno uticati na tržište poslova. Procenjeno je da će u naredne dve decenije ona zameniti 10p poslova. Istraživači u Međunarodnom monetarnom fondu analizirali su trideset država i predviđaju disproporcionalno veći gubitak poslova kod žena nego kod muškaraca. Razlog tome je što će automatizacija najviše uticati na pretežno administrativne poslove kojima se češće bave žene (Brussevich et al., 2018). 




\section{Odnos polova u informatici u Srbiji u odnosu na svet}	




\section{Odnos polova u informatici na Matematičkom fakultetu Univerziteta u Beogradu}
\label{matematicki}
Problem rodne jednakosti, tj. nejednakosti je deo kulturnog nasleđa i teško je uticati na tradicionalne poglede i podelu poslova na osnovu polova. Veliku odgovornost u tome da zainteresuju žene za IT oblast kao i da im omoguće kvalitetno obrazovanje imaju sve obrazovne institucije, a posebno Univerziteti. Bez toga, trenutna slika IT sektora teško će moći da se promeni i unapredi.
U ovom odeljku prikazaćemo podatke o odnosu polova na smeru Informatika na Matematičkom fakultetu Univerziteta u Beogradu u prethodnih šest godina. \\

\begin{table}[h!]
\begin{center}
\caption{Broj upisanih studenata muškog i ženskog pola u prethodnih šest godina}

\begin{tabular}{|c|c|c|c|} \hline
godina&broj upisanih muškog pola& broj upisanih ženskog pola & ukupan broj upisanih\\ \hline
2017&144&47&161\\ \hline
2018&102&54&156\\ \hline
2019&90&75&165\\ \hline
2022&104&56&160\\ \hline

\end{tabular}
\label{tab:tabela1}
\end{center}
\end{table}

\begin{tikzpicture}
\pie{70.8/M, 29.2/Ž}
\end{tikzpicture}

\begin{tikzpicture}
\pie{65.38/M, 34.62/Ž}
\end{tikzpicture}

\begin{tikzpicture}
\pie{54.55/M, 45.45/Ž}
\end{tikzpicture}

\begin{tikzpicture}
\pie{65/M, 35/Ž}
\end{tikzpicture}


\section{Zaključak}
\label{sec:zakljucak}

TODO


\addcontentsline{toc}{section}{Literatura}
\appendix
\bibliography{seminarski} 
\bibliographystyle{plain}

\appendix


\end{document}
